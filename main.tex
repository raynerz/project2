%Vorlage
\documentclass[12pt,a4paper]{scrartcl}
\usepackage[english]{babel} %Für die indirekte Angabe von Umlauten. Es müssen dann Umlaute wie folgt im Code angegeben werden: "a "o "u "s.

\usepackage[utf8]{inputenc}
%Math
\usepackage{amsmath, amsthm, amssymb}
\usepackage{braket}
\newcommand{\tens}[1]{% https://tex.stackexchange.com/questions/171788/always-have-the-ring-of-the-tensor-product-below-the-otimes -> Tensor Product
  \mathbin{\mathop{\otimes}\displaylimits_{#1}}%
}
%Page numbers
\usepackage{enumerate}
%Graphics
\usepackage{graphicx}
\usepackage{floatrow}
\graphicspath{{./images/}}
%Quantum circuits http://mirrors.ibiblio.org/CTAN/graphics/pgf/contrib/quantikz/quantikz.pdf
\usepackage{tikz}
\usetikzlibrary{quantikz}
\usepackage{lscape}
\usepackage{setspace}
\onehalfspacing
\usepackage{wrapfig}
\usepackage{hyperref}% für die Einbettung von Hyperlinks
\def\UrlBreaks{\do\/\do-}
\usepackage{multirow}
\usepackage{csquotes} %Quotations


% Margins
%\usepackage{geometry} % Document Margins
%\setlength{\topmargin}{0cm}
%\setlength{\parindent}{5mm}
%\setlength{\parskip}{2mm}
%\setlength{\evensidemargin}{0mm}
%\setlength{\oddsidemargin}{0cm}
\pagestyle{headings}



\begin{document}
\thispagestyle{empty}
\vspace*{-3cm}
\begin{center}
\large \textsc{Bern University of Applied Sciences}
\vspace{0.5cm}
\hrule
\vspace{4.5cm}
{\Large \textsc{Project II - BTI7302}}\\
{\large HS 2020/21}\\
\vspace{1cm}
{\Large \bfseries
Computer vision methods for detection of of early choroidal thickness changes in myopic Asian school children}\\
\vspace*{1cm}
{\large Tutor:  Prof. Dr. Tiziano Ronchetti}
\end{center}
\vspace*{1cm}

\begin{abstract}
\textbf{Abstract: }Retinal layers thickness measurement offers physicians a reliable method for diagnose and treatment of ocular and other diseases. The computerized implementation of this technique involves the utilisation of computer vision algorithms in order to detect and measure retinal layers from OCT scans. In this report, the authors explore these algorithms and implement the techniques in order to obtain a reliable set of algorithmically generated annotations in order to evaluate the application of machine learning for detecting retinal layers and accurately identifying thickness measurement changes with minimal human intervention. \\
\textbf{Keywords: } Ophthalmology, Retinal Layers, Machine Learning, Computer Vision, Data Analysis
\end{abstract}

\vspace{2cm}
\hspace*{5.2cm}
\parbox{8.2cm}

\begin{tabular}{ll}

Submitted by: & Emeline Liebeherr\\
& Rayner Oswaldo Däppen\\

Submission deadline: & Friday, January 30th, 2021


\end{tabular}

\newpage
\pagenumbering{Roman}
\tableofcontents
\newpage
\listoffigures
\listoftables

\newpage
\pagenumbering{arabic}
%Und nun kommen wir zur Arbeit und fangen an die Seiten mit Arabischen Zahlen zu zählen

\section{Introduction}\label{s:introduction}
The main objective of this paper is to provide an implementation of the most common algorithmic methods used in the field of computer vision for segmenting retinal inner layers obtained from OCT scans in order to produce a reliable set of annotations that can be used in a future work for developing a machine learning model that can accurately detect retinal thickness changes and provide physicians with reliable and factual information that can help in the early diagnosis of myopia. \\

The report is structured as follows. Section 2 presents an introduction to the medical background needed in order to understand the ocular globe structure and the importance of retinal layer thickness measurement from an ophtalmological perspective. Section 3 will offer an overview of the most common techniques used in the industry for carrying out retinal layer segmentation from OCT scans. Section 4 will offer the implementation and results of the methods discussed. Finally, section 5 and 6 will respectively offer a conclusion and an overview of the future work to be carried out by the authors within the project.  

\section{Medical Background}\label{s:medical_background}

\section{Technical Background}
\section{Results}
\section{Conclusion}
\section{Future Work}


\markboth{}{}

\newpage

\bibliographystyle{plain} % We choose the "plain" reference style
\bibliography{bibli} % Entries are in the "bibi.bib" file




\newpage
\thispagestyle{empty}
\markboth{}{}
  \normalsize
\begin{center}
\huge{\textbf{ Declaration of Independence}}\\[40mm]
\end{center}
\large
I affirm that the above work has been produced by myself without any unauthorized assistance and without the use of any other means than those indicated, and that I have marked as such all passages that have been taken literally or meaningfully from published or unpublished writings.\\[50mm]
Bienne, the \today

\newpage



\end{document}