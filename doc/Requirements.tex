%Vorlage
\documentclass[12pt,a4paper]{article}
\usepackage[english]{babel} %Für die indirekte Angabe von Umlauten. Es müssen dann Umlaute wie folgt im Code angegeben werden: "a "o "u "s.

\usepackage[utf8]{inputenc}
%dieses Paket ermöglicht uns, Umlaute im Text als solche eingeben zu können (Windows/Linux)

%\usepackage[applemac]{inputenc}
%Mac-Nutzer, die Ihre Latex-Dateien lokal kompilieren wollen, müssen dieses Paket aktivieren, um Umlaute im Code direkt angeben zu können.

\usepackage{amsmath, amsthm, amssymb}
\usepackage{enumerate}
\usepackage{graphicx}
\usepackage{lscape}
\usepackage{setspace}
\onehalfspacing
\usepackage{wrapfig}
\usepackage{hyperref}% für die Einbettung von Hyperlinks
\usepackage{multirow}
\usepackage[round]{natbib}


%\newtheorem{definition}{Definition}[section]
%\newtheorem{interview}{Interview}[section]
%\newtheorem{satz}{Satz}[section]
%\newtheorem{example}{Example}[section]
%\newtheorem{note}{Note}[section]
%\newtheorem{bibliography}{Bibliography}[section]

% Margins
\usepackage[left=25mm,right=25mm,bottom=20mm,top=25mm]{geometry} % Document Margins
\setlength{\topmargin}{0cm}
\setlength{\parindent}{5mm}
\setlength{\parskip}{2mm}
\setlength{\evensidemargin}{0mm}
\setlength{\oddsidemargin}{0cm}
%\pagestyle{headings}



\begin{document}
\thispagestyle{empty}
\vspace*{-3cm}
\begin{center}
\large \textsc{Bern University of Applied Sciences}
\vspace{0.5cm}
\hrule
\vspace{5.5cm}
{\Large \textsc{Documentation\\
(BTI7302-Projet 2)}}\\
{\large HS 2020/21}\\
\vspace{1cm}
{\Large \bf
!!! [NOM DU PROJET] !!!}\\
\vspace*{1cm}
{\large Referent:  Prof. Dr. Ronchetti Tiziano}
\end{center}
\vspace*{5cm}
{\large

\hspace*{7cm}
\parbox{8.2cm}
{
\begin{tabular}{ll}
Submitted by & Rayner Oswaldo Däppen && Emeline Lieberherr\\

Submission deadline: & [to de defined]

\end{tabular}}}

\newpage
\pagenumbering{Roman}
\tableofcontents

\newpage
\pagenumbering{arabic}
%Und nun kommen wir zur Arbeit und fangen an die Seiten mit Arabischen Zahlen zu zählen
\section{Project overview}\label{Project overview}

\section{Scope}\label{Scope}

\section{Goals}\label{Goals}

\section{Requirements}\label{Requirements}


\begin{itemize}
  \item Put in place an enviroment to run deep learning models
  \item Organise the work environement and define a structure of work
  \item Prepare the data to be used by the model
  \item Find the lines of the retina and chroid to define the XXX
  \item Extract a surface form the lines found to define the XXX width
\end{itemize}


\newpage
\begin{thebibliography}{9}

\bibitem{OCT_kaggle}
Retinal OCT Images (optical coherence tomography),
\\\texttt{https://www.kaggle.com/paultimothymooney/kermany2018?}

\bibitem{OCT_dataset}
Eye OCT Datasets,
\\\texttt{https://www.kaggle.com/kmader/eye-oct-datasets}

\bibitem{gitlab_gitkraken_doc}
Gitlab - GitKraken documentation,
\\\texttt{https://support.gitkraken.com/integrations/gitlab/}

\bibitem{Edge_detectors}
The Edge Detectors Suitable for Retinal OCT Image Segmentation,
\\\texttt{http://downloads.hindawi.com/journals/jhe/2017/3978410.pdf}

\end{thebibliography}



\end{document}
