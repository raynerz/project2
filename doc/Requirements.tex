%Vorlage
\documentclass[12pt,a4paper]{scrartcl}
\usepackage[english]{babel} %Für die indirekte Angabe von Umlauten. Es müssen dann Umlaute wie folgt im Code angegeben werden: "a "o "u "s.

\usepackage[utf8]{inputenc}
%Math
\usepackage{amsmath, amsthm, amssymb}
\usepackage{braket}
\newcommand{\tens}[1]{% https://tex.stackexchange.com/questions/171788/always-have-the-ring-of-the-tensor-product-below-the-otimes -> Tensor Product
  \mathbin{\mathop{\otimes}\displaylimits_{#1}}%
}
%Page numbers
\usepackage{enumerate}
%Graphics
\usepackage{graphicx}
\usepackage{floatrow}
\graphicspath{{./images/}}
%Quantum circuits http://mirrors.ibiblio.org/CTAN/graphics/pgf/contrib/quantikz/quantikz.pdf
\usepackage{tikz}
\usetikzlibrary{quantikz}
\usepackage{lscape}
\usepackage{setspace}
\onehalfspacing
\usepackage{wrapfig}
\usepackage{hyperref}% für die Einbettung von Hyperlinks
\def\UrlBreaks{\do\/\do-}
\usepackage{multirow}
\usepackage{csquotes} %Quotations


% Margins
%\usepackage{geometry} % Document Margins
%\setlength{\topmargin}{0cm}
%\setlength{\parindent}{5mm}
%\setlength{\parskip}{2mm}
%\setlength{\evensidemargin}{0mm}
%\setlength{\oddsidemargin}{0cm}
\pagestyle{headings}

%Code

\usepackage{listings}
\usepackage{xcolor}

\definecolor{codegreen}{rgb}{0,0.6,0}
\definecolor{codegray}{rgb}{0.5,0.5,0.5}
\definecolor{codepurple}{rgb}{0.58,0,0.82}
\definecolor{backcolour}{rgb}{0.95,0.95,0.92}

\lstdefinestyle{mystyle}{
    backgroundcolor=\color{backcolour},   
    commentstyle=\color{codegreen},
    keywordstyle=\color{magenta},
    numberstyle=\tiny\color{codegray},
    stringstyle=\color{codepurple},
    basicstyle=\ttfamily\footnotesize,
    breakatwhitespace=false,         
    breaklines=true,                 
    captionpos=b,                    
    keepspaces=true,                 
    numbers=left,                    
    numbersep=5pt,                  
    showspaces=false,                
    showstringspaces=false,
    showtabs=false,                  
    tabsize=2
}
\lstset{style=mystyle}



\begin{document}
\thispagestyle{empty}
\vspace*{-3cm}
\begin{center}
\large \textsc{Bern University of Applied Sciences}
\vspace{0.5cm}
\hrule
\vspace{5.5cm}
{\Large \textsc{Written Report\\
(BTI7311-Informatik Seminar)}}\\
{\large HS 2020/21}\\
\vspace{1cm}
{\Large \bfseries
Computer vision methods for detection of early choroidal thickness changes in myopic Asian school children}\\
Requirements set
\vspace*{1cm}
{\large Referent:  Prof. Dr. Tiziano Ronchetti}
\end{center}
\vspace*{1cm}

\begin{abstract}

\end{abstract}

\vspace{2cm}
\hspace*{5.2cm}
\parbox{8.2cm}

\begin{tabular}{ll}

Submitted by: & Emeline Liebeherr 
& Rayner Oswaldo Däppen\\

Submission deadline: & Friday, January 8th, 2021


\end{tabular}
\newpage
\pagenumbering{Roman}
\tableofcontents

\newpage
\pagenumbering{arabic}
%Und nun kommen wir zur Arbeit und fangen an die Seiten mit Arabischen Zahlen zu zählen
\section{Project overview}\label{Project overview}

\section{Scope}\label{Scope}

\section{Goals}\label{Goals}

\section{Requirements}\label{Requirements}


\begin{itemize}
  \item Put in place an enviroment to run deep learning models
  \item Organise the work environement and define a structure of work
  \item Prepare the data to be used by the model
  \item Find the lines of the retina and chroid to define the XXX
  \item Extract a surface form the lines found to define the XXX width
\end{itemize}


\newpage
\begin{thebibliography}{9}

\bibitem{OCT_kaggle}
Retinal OCT Images (optical coherence tomography),
\\\texttt{https://www.kaggle.com/paultimothymooney/kermany2018?}

\bibitem{OCT_dataset}
Eye OCT Datasets,
\\\texttt{https://www.kaggle.com/kmader/eye-oct-datasets}

\bibitem{gitlab_gitkraken_doc}
Gitlab - GitKraken documentation,
\\\texttt{https://support.gitkraken.com/integrations/gitlab/}

\bibitem{Edge_detectors}
The Edge Detectors Suitable for Retinal OCT Image Segmentation,
\\\texttt{http://downloads.hindawi.com/journals/jhe/2017/3978410.pdf}

\end{thebibliography}



\end{document}
